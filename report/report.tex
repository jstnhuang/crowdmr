\documentclass{article} % For LaTeX2e
\usepackage{nips13submit_e, times, graphicx}

\title{Experiences implementing peer-to-peer MapReduce in the browser}

\author{Justin Huang \And Jijiang Yan}

\newcommand{\fix}{\marginpar{FIX}}
\newcommand{\new}{\marginpar{NEW}}

\nipsfinalcopy

\begin{document}

\maketitle

\begin{abstract}
MapReduce~\cite{dean2008mapreduce} is a popular tool for processing large
amounts of data. However, the average person typically doesn't have easy access
to a MapReduce implementation. This paper presents Crowd MapReduce, which allows
jobs to be created and run through a browser-based interface. The computation is
distributed across volunteers who visit a given URL, in a peer-to-peer way. We
found that our system did not deliver good performance, and discuss the
challenges involved in implementing our system.
\end{abstract}

\section{Introduction}
The typical person can run a MapReduce job in one of two ways. Either they have
access to a server cluster with some MapReduce implementation already installed
and configured, or they pay for access to a system running on a cloud provider's
servers. Both of these methods can be difficult for researchers who want to use
MapReduce, but have neither the expertise nor the resources to set it up. At the
same time, there is a trend towards crowdsourcing, where people are
increasingly willing to donate time and resources for good causes.

In response, we have designed and implemented Crowd MapReduce, which allows
people to create and run MapReduce jobs through a simple browser interface.
Instead of using dedicated servers, the computation and data are distributed in
a peer-to-peer fashion across volunteers who visit a particular URL.
Additionally, the system is implemented as a client-side application, meaning
that running the system just requires a web server to serve up static files.

\subsection{Design goals and assumptions}
The main goal of the system is to be extremely simple to use. It should enable
people to run MapReduce jobs without owning or having access to other computers
in the cloud. A secondary goal of the system is to test the limits of modern web
technologies, and see how close they are to providing a useful platform for
applications such as Crowd MapReduce.

One assumption we are making is that the job creator is reliable, and will keep
a job tracking webpage open until the job is complete. On the other hand, we
assume that volunteers are not as reliable, and may decide to disconnect at any
time.

We also assume that the user's data can be represented as text. We represent all
data as strings or arrays of strings for performance reasons.

Additionally, since the job creator is sending volunteers code to execute and
volunteers are sending data back, security may be a concern. However, we assume
that neither the job tracker nor the volunteers are malicious, and will run our
system as given.

\section{Previous work}
A similar idea was discussed in 2009 by Ilya Grigorik on his
blog~\cite{grigorik2009}. However, he only discussed the idea of such a system
without fully implementing it.

In 2010, Ryza and Wall provided an implementation of browser-based MapReduce for
a class project~\cite{mrjs}.
Langhans, Wieser, and Bry presented a similar
system~\cite{langhans2013crowdsourcing} to that of Ryza and Wall. Both systems use a centralized server, where clients request work over HTTP. This setup is
vulnerable to server failure -- if the server goes down, every job running on
the server will fail. In contrast, our system is run completely in the browser.
This means that if a job creator's browser crashes, then their job will be lost.
However, no one else's job will be affected.

\section{Implementation}
In this section, we will describe our implementation of Crowd MapReduce in
detail, by following the creation and execution of an example job.

\subsection{Job creation}
When someone wants to create a job, they go to the job creation page, shown in
figure~\ref{creationpage}. The user inputs Javascript function bodies for the
mapper and reducer, conforming to the API given in figure~\ref{api}.

\begin{figure}[h]
  \centering
    \includegraphics[width=0.75\textwidth]{create}
    \caption{A screenshot of the job creation page.}
    \label{creationpage}
\end{figure}

\begin{figure}[h]
  \begin{tabular}{| l | p{5.5cm} | p{6cm} |}
    \hline
    Function & Input & Output \\ \hline
    mapper & An array of strings & An array of key/value pairs as strings, each
    with format:
    key$<$TAB$>$value \\ \hline
    reducer & A key and an array of values, as strings & A reduced array
    of string values
    \\
    \hline
  \end{tabular}
  \caption{The API for the mapper and reducer functions.}
  \label{api}
\end{figure}

Next, the user must upload their data. Because we assume the user has no access
to cloud storage, the data is instead copied to a virtual filesystem provided by
the browser. This is only possible for browsers implementing the HTML 5
FileSystem API~\cite{html5file}. The FileSystem API allows applications to store
an arbitrary amount of data in the virtual filesystem, as long as the user
approves it.

We assume that the user's data is already broken into multiple files, such that
each file will be the input to a mapper. Hence, the number of mappers is equal
to the number of files. Our application could possibly split the user's data
automatically, given a maximum size for each map task. However, we chose not to
do this due to time constraints. This functionality is already provided by the
Unix \texttt{split} utility.

Finally, the user specifies the number of reduce partitions for the job. The
concept of reduce partitions in our system is similar to that of the Google
MapReduce implementation. More reduce partitions leads to greater parallelism,
but also leads to more output files, which the user must be prepared to combine
on their own.

\subsection{Job tracker}
Once the job has been created, the user is taken to a job tracker page, shown in
figure~\ref{maprunning}.

\begin{figure}[h]
  \centering
    \includegraphics[width=0.75\textwidth]{maprunning}
    \caption{A screenshot of the job tracker page, showing a job in progress.}
    \label{maprunning}
\end{figure}

The job tracker provides several pieces of information:
\begin{itemize}
  \item The URL to share to volunteers.
  \item The number of volunteers who are running jobs.
  \item The progress of the job, expressed both as an overall percentage, and as
  a breakdown of the tasks that are either waiting to run, running, or complete.
  \item Download links for the output data when the job completes.
\end{itemize}

In order to establish a peer-to-peer connection between the job tracker and the
volunteers, we use another recent browser technology, WebRTC~\cite{webrtc}.
WebRTC is a browser API which allows peers to send arbitrary data to each other.
In our implementation, we used the PeerJS library~\cite{peerjs}, which
simplifies the WebRTC APIs.

The job tracker maintains two primary data structures. The first data structure
keeps track of what tasks need to be done, which tasks are running, and which
are complete. Each task specifies the location of a file on the virtual
filesystem, which contains the input data for the task. The second data
structure is a list of clients who have connected, as well as what task they
are assigned to.

When a client connects, an idle task is assigned to a client. The job tracker
sends the code and the task, which contains the location of the data, to the
client. In the normal case, the client doesn't have the data, in which case it
requests the data from the job tracker. The job tracker then loads the input
data from the virtual filesystem into memory, and sends it to the client.

If a client disconnects, the job tracker looks up which task they were assigned
to, and moves the task back into the idle queue. Otherwise, the client responds
with the results. For both map and reduce tasks, the results are in the form of
key/value pairs. The job tracker writes each result to one of the reduce
partitions on the filesystem based on the hash of the key.

The reduce phase does not start until all the map tasks are done. For a
sufficiently large dataset, the number of reduce partitions that were actually
written to will be the same as the number of reduce partitions the job creator
specified. When all the reduce tasks are complete, the job tracker provides
download links for the output files from the virtual filesystem.

\subsection{Job client}
The URL for a job client page contains the unique job ID necessary for the
client to connect to the job tracker. When it is a assigned a map task, it
simply evaluates the mapper code as a Javascript function, and passes in the
array of data given by the job tracker. It then returns the results of the
mapper back to the job tracker.

When the client processes a reduce task, it takes the additional step of sorting
the data and grouping the values by key. The key and the array of values are
passed to the reducer, which returns the reduced array of values. When finished,
the client sends the data back to the job tracker as a flattened list of
key/value pairs.

\subsection{Local clients}
The job creator may want to run the job on their local machine. In that case, it
doesn't make sense to send data over the network. When the job tracker sends a
task to the client, the client checks the browser's filesystem to see if it has
the data necessary for the task. If it does have the data, the client realizes
it is running in the same browser as the job creator. We call this a local
client.

The local client behaves similarly to the client, except that it loads data and
writes results directly to the filesystem instead of sending and receiving it
over the network to the job tracker. However, the local client and the job
tracker are executing in a separate processes, so they still communicate tasks
and statuses over the network.

\section{Experiments}
We evaluated the performance of our system, varying several variables:
\begin{itemize}
  \item The number of clients
  \item The size of the input file chunks
  \item The number of reducers
  \item Whether the clients are local or not
\end{itemize}

\subsection{Performance using only local clients}


\section{Design alternatives and future work}
Based on our results, it appears that the major limiting factor to the
performance of our system was the time taken to transport data between hosts.
However, part of this difference is simply due to different usage patterns.

In a typical MapReduce system, such as the one described by Dean and Ghemawat
\cite{dean2008mapreduce} the data is uploaded and distributed amongst the
worker nodes before the job starts. Additionally, the output data stays on the
worker nodes. This makes sense if many MapReduce jobs will be chained together,
the output of one becoming the input of another. In contrast, our system has a
``single use'' design, where the of sending input data to and downloading
output data from the worker nodes is included with every job.

A typical MapReduce system stores data on the worker nodes, so it can also
leverage the fact that reduce workers may already have some of the intermediate
data, whereas in our system, reduce workers must get all of the intermediate
data from the job tracker.

If, in our system, we could assume that volunteers could store files for long
periods of time, then we could design our system to more closely match a typical
MapReduce system. This would allow us to amortize the cost of uploading and
downloading data over multiple, chained jobs, and allow reduce workers to
potentially download less data.

Another design alternative is to assume that the job creator has access to some
kind of distributed cloud storage. This would reduce the job tracker's
bandwidth usage, since it would only send and receive control signals instead
of data. It would also allow us to amortize the cost of multiple jobs. However,
it would still require reduce workers to download the full amount of
intermediate data.

Another general area for future work would be to look into the use of WebWorkers
to handle filesystem and network operations. Because of the Javascript
concurrency model, every filesystem and network operation we perform is
blocking. WebWorkers are another relatively new browser technology that allow
code to be run in separate, heavyweight threads. This could potentially allow us
to make expensive operations non-blocking.

\section{Conclusion}
In this paper, we presented Crowd MapReduce, a MapReduce implementation that
allows users to run jobs in their browser with the help of other peers on the
network. Our system requires almost no server support because the application is
run entirely on the client side. We also evaluated the performance of our system and
discussed the issues we faced in building it.

\bibliographystyle{unsrt}
\bibliography{report}

\end{document}